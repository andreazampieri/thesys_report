\documentclass{article}

\newcommand\tab[1][1cm]{\hspace*{#1}}
\usepackage{scrextend}
\makeatletter
\let\@addmarginORIG\@addmargin
\renewcommand*\@addmargin{%
	\vspace{-\bigskipamount}
	\@addmarginORIG}
\makeatother

\author{Andrea Zampieri}
\title{Network Inference from Node Embedding:\newline Deep Autoencoding}


\begin{document}
	\maketitle
	\newpage
	\section{Introduction} 
		I started from the Zekarias	T. Kefato's paper: \textit{DeepInfer: Diffusion Network Inference through Representation Learning}.\\
		The problem tackled in this paper (as described by its title) is inferring a network starting from series of interactions between the elements in play.
		The goal is to obtain a reconstruction as precise as possible of the actual \textbf{Interaction Network} without using any kind of exact information on the actual graph.\\ 
		\\
		\textbf{Note:} for simplicity and immediacy sake, during the explanation I will make references to a specific example in order to show an instance of the problem:\\
		
		\begin{addmargin}[2em]{0em}
			\textbf{\textit{Twitter}}: the interactions observed are formed by:\\
				\begin{addmargin}[2em]{0em}
					- \textit{contagious element}: the ones taken in consideration are the \textit{hashtags} \\
					- \textit{infection spreading}: each contagiant has a list of user infected (with the timestamp). This indicates which users used that hashtag.
				\end{addmargin}
		\end{addmargin}
	
	\section{Reproduction of the SoA Tecnique}
		\subsection{Brief Overview}
			The State of Art technique has precise, consequential steps:
			\begin{itemize}
				\item Cascade Linearization
				\item Node embedding
				\item Top Pairs fetching
				\item Inference evaluation
			\end{itemize}
		\subsection{Description}
			The desired input formatting is a list of sequences of interaction, the contagiant element and the time between each infection isn't relevant. This kind of lists are called cascades.\\
			In order to obtain this kind of representation - for each hashtag cascade - it's necessary to sort each user for ascending timestamp, and once did so, stripe off the time values, since aren't required for the computations.\\
			\paragraph{Example}
				\begin{addmargin}[1em]{0em}
					$$c_{j}: [(u_{1},t_{1}), ..., (u_{n},t_{n})]$$
					\\
					where $u_{i}$ is the id of the user and $t_{i}$ its timestamp relative to the cascade $c_{j}$ represented by its hashtag.\\
					The representation becomes: $[u_{k},...,u_{l}]$ where the users $u$ are sorted by ascending timestamp. \textbf{Note:} the hashtag and the timestamps are stripped off because they aren't relevant to the goal of network inference.
				\end{addmargin}
		\subsection{Node Embedding}
			\textit{Node Embedding} states for a different representation of the nodes of the graph (in our case, the users).
\end{document}